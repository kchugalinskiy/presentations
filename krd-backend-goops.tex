\documentclass[mathserif,serif]{beamer}
\usetheme{Warsaw}
\usepackage[utf8]{inputenc}
\usepackage[russian]{babel}
\usepackage{listings}

\title[DevOps глазами разработчика]
{DevOps глазами разработчика}
\subtitle{Или о чем молчат операторы}
\date
{}

\begin{document}
\frame{\titlepage}

\begin{frame}
	\frametitle{Что такое DevOps}
	\includegraphics[width=\linewidth]{dev_ops_juinor.png}
\end{frame}

\begin{frame}
	\frametitle{Управление инфраструктурой}
	\begin{itemize}
		\item{bash/perl скрипты}
		\item{Ansible}
		\item{Terraform}
		\item{Инфраструктурные сервисы}
	\end{itemize}
\end{frame}

\begin{frame}
	\frametitle{bash/perl-скрипты}
	\begin{columns}[T]
		\begin{column}{.5\linewidth}
			\begin{minipage}[c][\textheight][c]{\linewidth}
          			Плюсы
				\begin{itemize}
					\item{Низкий порог вхождения}
					\item{Простота использования}
				\end{itemize}
			\end{minipage}
		\end{column}
		\begin{column}{.5\linewidth}
			\begin{minipage}[c][\textheight][c]{\linewidth}
          			Минусы
				\begin{itemize}
					\item{Нет транзакций}
					\item{Нет состояния}
					\item{Трудности поддержки}
				\end{itemize}
			\end{minipage}
		\end{column}
	\end{columns}
\end{frame}

\begin{frame}
	\frametitle{Ansible}
	\begin{columns}[T]
		\begin{column}{.5\linewidth}
			\begin{minipage}[c][\textheight][c]{\linewidth}
          			Плюсы
				\begin{itemize}
					\item{Infrastructure as Code (Императивный подход)}
					\item{Простота использования}
					\item{.retry-файлы}
				\end{itemize}
			\end{minipage}
		\end{column}
		\begin{column}{.5\linewidth}
			\begin{minipage}[c][\textheight][c]{\linewidth}
          			Минусы
				\begin{itemize}
					\item{Нет полноценного состояния}
					\item{Технологические ограничения автоматизации}
				\end{itemize}
			\end{minipage}
		\end{column}
	\end{columns}
\end{frame}

\begin{frame}
	\frametitle{Terraform}
	\begin{columns}[T]
		\begin{column}{.5\linewidth}
			\begin{minipage}[c][\textheight][c]{\linewidth}
          			Плюсы
				\begin{itemize}
					\item{Infrastructure as Code (декларативный подход)}
					\item{Полноценное состояние}
					\item{Совместная разработка}
					\item{Модульность}
				\end{itemize}
			\end{minipage}
		\end{column}
		\begin{column}{.5\linewidth}
			\begin{minipage}[c][\textheight][c]{\linewidth}
          			Минусы
				\begin{itemize}
					\item{Ограниченные сферы применения}
					\item{Нельзя писать полноценный активный код}
					\item{Высокий порог входа}
				\end{itemize}
			\end{minipage}
		\end{column}
	\end{columns}
\end{frame}

\begin{frame}
	\frametitle{Инфраструктурные сервисы}
	\begin{columns}[T]
		\begin{column}{.5\linewidth}
			\begin{minipage}[c][\textheight][c]{\linewidth}
          			Плюсы
				\begin{itemize}
					\item{Полноценное состояние, все зависит от нас}
					\item{Интеграция со всеми мыслимыми библиотеками}
					\item{Полноценный код, управляющий инфраструктурой}
				\end{itemize}
			\end{minipage}
		\end{column}
		\begin{column}{.5\linewidth}
			\begin{minipage}[c][\textheight][c]{\linewidth}
          			Минусы
				\begin{itemize}
					\item{ОЧЕНЬ Высокий порог входа, нужны разработчики}
					\item{Шанс изобрести велосипед}
				\end{itemize}
			\end{minipage}
		\end{column}
	\end{columns}
\end{frame}

\begin{frame}
	\frametitle{DevOps - не человек}
	\includegraphics[width=\linewidth]{true-devops.jpg}
\end{frame}

\end{document}