\documentclass[mathserif,serif]{beamer}
\usetheme{Warsaw}
\usepackage[utf8]{inputenc}
\usepackage[russian]{babel}
\usepackage{listings}

\title[DevOps глазами разработчика]
{DevOps глазами разработчика}
\subtitle{Или о чем молчат операторы}
\date
{}

\begin{document}
\frame{\titlepage}

\begin{frame}
	\frametitle{Что такое DevOps}
	\includegraphics[width=\linewidth]{dev_ops_juinor.png}
\end{frame}

\begin{frame}
	\frametitle{Управление инфраструктурой}
	\begin{itemize}
		\item{bash/perl скрипты}
		\item{Ansible}
		\item{Terraform}
		\item{Инфраструктурные сервисы}
	\end{itemize}
\end{frame}

\begin{frame}
	\frametitle{bash/perl-скрипты}
	\begin{columns}[T]
		\begin{column}{.5\linewidth}
			\begin{minipage}[c][\textheight][c]{\linewidth}
          			Плюсы
				\begin{itemize}
					\item{Низкий порог вхождения}
					\item{Простота использования}
				\end{itemize}
			\end{minipage}
		\end{column}
		\begin{column}{.5\linewidth}
			\begin{minipage}[c][\textheight][c]{\linewidth}
          			Минусы
				\begin{itemize}
					\item{Нет транзакций}
					\item{Нет состояния}
					\item{Трудности поддержки}
				\end{itemize}
			\end{minipage}
		\end{column}
	\end{columns}
\end{frame}

\begin{frame}
	\frametitle{Ansible}
	\begin{columns}[T]
		\begin{column}{.5\linewidth}
			\begin{minipage}[c][\textheight][c]{\linewidth}
          			Плюсы
				\begin{itemize}
					\item{Infrastructure as Code (Императивный подход)}
					\item{Простота использования}
					\item{.retry-файлы}
				\end{itemize}
			\end{minipage}
		\end{column}
		\begin{column}{.5\linewidth}
			\begin{minipage}[c][\textheight][c]{\linewidth}
          			Минусы
				\begin{itemize}
					\item{Нет полноценного состояния}
					\item{Технологические ограничения автоматизации}
				\end{itemize}
			\end{minipage}
		\end{column}
	\end{columns}
\end{frame}

\begin{frame}
	\frametitle{Terraform}
	\begin{columns}[T]
		\begin{column}{.5\linewidth}
			\begin{minipage}[c][\textheight][c]{\linewidth}
          			Плюсы
				\begin{itemize}
					\item{Infrastructure as Code (декларативный подход)}
					\item{Полноценное состояние}
					\item{Совместная разработка}
					\item{Модульность}
				\end{itemize}
			\end{minipage}
		\end{column}
		\begin{column}{.5\linewidth}
			\begin{minipage}[c][\textheight][c]{\linewidth}
          			Минусы
				\begin{itemize}
					\item{Ограниченные сферы применения}
					\item{Нельзя писать полноценный активный код}
					\item{Высокий порог входа}
				\end{itemize}
			\end{minipage}
		\end{column}
	\end{columns}
\end{frame}

\begin{frame}
	\frametitle{Инфраструктурные сервисы}
	\begin{columns}[T]
		\begin{column}{.5\linewidth}
			\begin{minipage}[c][\textheight][c]{\linewidth}
          			Плюсы
				\begin{itemize}
					\item{Полноценное состояние, все зависит от нас}
					\item{Интеграция со всеми мыслимыми библиотеками}
					\item{Полноценный код, управляющий инфраструктурой}
				\end{itemize}
			\end{minipage}
		\end{column}
		\begin{column}{.5\linewidth}
			\begin{minipage}[c][\textheight][c]{\linewidth}
          			Минусы
				\begin{itemize}
					\item{ОЧЕНЬ Высокий порог входа, нужны разработчики}
					\item{Шанс изобрести велосипед}
				\end{itemize}
			\end{minipage}
		\end{column}
	\end{columns}
\end{frame}

\begin{frame}
	\frametitle{DevOps - не человек, DevOps - методология}
	\includegraphics[width=\linewidth]{true-devops.jpg}
\end{frame}

\begin{frame}
	\frametitle{Sample terraform repo}
	\includegraphics[width=0.7\linewidth]{qrcode.png}
\end{frame}

\begin{frame}
	\frametitle{Что такое Terraform}
				\begin{itemize}
					\item{Создание ресурсов провайдера}
					\item{Сохранение созданных ресурсов в state-файле}
					\item{Обновление и удаление ресурсов}
					\item{Различные провайдеры (AWS, Vault, Kubernetes, Consul)}
					\item{Богатый набор свободных модулей}
				\end{itemize}
\end{frame}

\begin{frame}[fragile]
	\frametitle{Что такое Terraform - пример}
	\begin{verbatim}
resource "kubernetes_namespace" "example" {
  metadata {
    annotations {
      name = "example-annotation"
    }

    labels {
      mylabel = "label-value"
    }

    name = "terraform-example-namespace"
  }
}
	\end{verbatim}
\end{frame}


\begin{frame}[fragile]
	\frametitle{Наивный проект}
	\begin{verbatim}
$ ls -la
-rw-r--r-- Oct  3 17:18 asg.tf
-rw-r--r-- Oct  3 17:18 database-cassandra.tf
-rw-r--r-- Oct  3 17:18 database-rds.tf
-rw-r--r-- Oct  3 17:18 elb.tf
-rw-r--r-- Oct  3 17:18 iam.tf
-rw-r--r-- Oct  3 17:18 keys.tf
-rw-r--r-- Oct  3 17:18 main.tf
-rw-r--r-- Oct  3 17:18 Makefile
-rw-r--r-- Oct  3 17:18 network.tf
-rw-r--r-- Oct  3 17:18 policy.json
-rw-r--r-- Oct  3 17:18 README.md
-rw-r--r-- Oct  3 17:18 route53.tf
-rw-r--r-- Oct  3 17:18 s3.tf
	\end{verbatim}
\end{frame}

\begin{frame}
	\frametitle{Наивный проект}
	\begin{columns}[T]
		\begin{column}{.5\linewidth}
			\begin{minipage}[c][\textheight][c]{\linewidth}
          			Преимущества
				\begin{itemize}
					\item{Ресурсо-ориентированность}
					\item{Все перед глазами}
					\item{Простота управления ресурсами}
				\end{itemize}
			\end{minipage}
		\end{column}
		\begin{column}{.5\linewidth}
			\begin{minipage}[c][\textheight][c]{\linewidth}
          			Недостатки
				\begin{itemize}
					\item{Плохая масштабируемость}
					\item{Дублирование кода}
					\item{Низкая абстракция}
				\end{itemize}
			\end{minipage}
		\end{column}
	\end{columns}
\end{frame}

\begin{frame}[fragile]
	\frametitle{Менее наивный проект}
	\begin{verbatim}
|-- common
|-- implementation
|   |-- .......
|-- dev
|   |-- blue
|   |-- green
|   |-- switch
|   |-- vpc
|-- prod
    |-- blue
    |-- green
    |-- switch
    |-- vpc
	\end{verbatim}
\end{frame}

\begin{frame}
	\frametitle{Менее наивный проект}
	\begin{columns}[T]
		\begin{column}{.5\linewidth}
			\begin{minipage}[c][\textheight][c]{\linewidth}
          			Преимущества
				\begin{itemize}
					\item{Отсутствие дублирования кода}
					\item{Масштабируемость}
					\item{Небольшой риск потери состояния}
				\end{itemize}
			\end{minipage}
		\end{column}
		\begin{column}{.5\linewidth}
			\begin{minipage}[c][\textheight][c]{\linewidth}
          			Недостатки
				\begin{itemize}
					\item{Неудобство в управлении}
					\item{Сложность передачи переменных}
				\end{itemize}
			\end{minipage}
		\end{column}
	\end{columns}
\end{frame}

\begin{frame}[fragile]
	\frametitle{Навороченный проект}
	\begin{verbatim}
|-- bastion
|-- consul
|   |-- getter
|-- keys
|   |-- getter
|-- network
|   |-- getter
|-- peering
|-- transport
|-- vault
|-- zone
    |-- getter
	\end{verbatim}
\end{frame}

\begin{frame}
	\frametitle{Навороченный проект}
	\begin{columns}[T]
		\begin{column}{.5\linewidth}
			\begin{minipage}[c][\textheight][c]{\linewidth}
          			Преимущества
				\begin{itemize}
					\item{Все среды уместились на слайде}
					\item{Максимальная изоляция}
					\item{Блочно-модульная структура}
					\item{Абсолютная унификация сред}
				\end{itemize}
			\end{minipage}
		\end{column}
		\begin{column}{.5\linewidth}
			\begin{minipage}[c][\textheight][c]{\linewidth}
          			Недостатки
				\begin{itemize}
					\item{Сложное переключение контекста}
					\item{Сложное управление ресурсами}
					\item{getter-ы часто приходят в негодность}
				\end{itemize}
			\end{minipage}
		\end{column}
	\end{columns}
\end{frame}

\begin{frame}
	\frametitle{Структурное программирование}
	\begin{itemize}
		\item{Функция = модуль}
		\item{Вызов функции = использование модуля}
		\item{null\_resource = цикл}
		\item{locals = переменные}
		\item{map = структуры}
		\item{Свобода = рабство}
	\end{itemize}
\end{frame}

\begin{frame}
	\frametitle{Изоляция на уровне интерфейса}
	Порядок действий при добавлении нового слоя
	\begin{itemize}
		\item{Во внешнем репозитории создается модуль с UT и самодостаточной логикой}
		\item{В созданном модуле максимально используются community модули терраформа}
		\item{Создается новый слой в репозитории tf продукта}
		\item{Ничего в слое нельзя создавать кроме самого внешнего модуля с аргументами из среды окружения}
		\item{С помощью direnv переключаем среды с одной на другую и последовательно создаем новый слой}
	\end{itemize}
\end{frame}

\begin{frame}[fragile]
	\frametitle{getter модуля}
	Изоляция логики получения результатов работы модуля от самого модуля
	\begin{verbatim}
module "consul-getter" {
  source             = "../consul/getter"
  region             = "${var.region}"
  country            = "${var.country}"
  stage              = "${var.stage}"
  internal_zone_name = "${var.internal_zone_name}"
}
	\end{verbatim}
\end{frame}

\begin{frame}
	\frametitle{Выводы из моего личного опыта}
	\begin{itemize}
		\item{Ops-ы плохо структурируют код}
		\item{Ops-ы не могут в абстракцию}
		\item{Разработчики хорошо продумывают развитие системы}
		\item{Инфраструктурные сервисы сами операторы не напишут}
		\item{Ключ к успеху лежит во взаимодействии}
	\end{itemize}
\end{frame}

\begin{frame}
	\frametitle{Вопросы, сэр?}
	\includegraphics[width=\linewidth]{theend.jpg}
\end{frame}

\end{document}